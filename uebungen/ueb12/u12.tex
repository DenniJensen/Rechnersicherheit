\documentclass{scrartcl}

\usepackage{graphicx} % Required for the inclusion of images

\renewcommand{\labelenumi}{\alph{enumi}.}
% deutsche Sprache und Umlaute
\usepackage[utf8x]{inputenc}
\usepackage[ngerman]{babel}


%BibTex verweise
\usepackage{cite}

% für url pfade
\usepackage[hyphens]{url}

% setzt beim neuen Absatz das erste Wort 4 mm nach rechts \\ kleine Abstand vom Rand
\setlength{\parindent}{4mm}

%für boxen
\usepackage{pifont,mdframed}

\usepackage{microtype}

% hack für scr paket für Indexierung
\usepackage{scrhack}

% erzwingen der Positionierung
\usepackage{float}
\restylefloat{figure}

% Verbindungen vom Index, Tabellen und Abbildungen
\usepackage{hyperref}

%Tiefe des Inhaltsverzeichnis
\setcounter{tocdepth}{5}

%für Unterstriche
\usepackage{underscore}

% Beschriftung
\usepackage{caption}
\captionsetup{%
  font=small,
  labelfont=bf,
}

\usepackage{fancybox}

%\geometry{showframe}% for debugging purposes -- displays the margins

\usepackage{amsmath}
\usepackage{amssymb}
\usepackage{geometry}
\geometry{verbose,a4paper,tmargin=20mm,bmargin=25mm,lmargin=15mm,rmargin=20mm}
% Set up the images/graphics package

% The following package makes prettier tables.  We're all about the bling!
\usepackage{booktabs}

% The units package provides nice, non-stacked fractions and better spacing
% for units.
\usepackage{units}

% The fancyvrb package lets us customize the formatting of verbatim
% environments.  We use a slightly smaller font.
\usepackage{fancyvrb}
\fvset{fontsize=\normalsize}

% Small sections of multiple columns
\usepackage{multicol}

% Provides paragraphs of dummy text
\usepackage{lipsum}

% listing
\usepackage{listings}
\usepackage{color}
\definecolor{grey}{rgb}{0.4,0.4,0.4}
\definecolor{darkblue}{rgb}{0.0,0.0,0.6}
\definecolor{cyan}{rgb}{0.0,0.6,0.6}

\lstset{
  basicstyle=\small,
  columns=fullflexible,
  showstringspaces=false,
  commentstyle=\color{gray}\upshape
}

\lstdefinelanguage{XML} {
  morestring=[b]",
  morestring=[s]{>}{<},
  morecomment=[s]{<?}{?>},
  stringstyle=\color{black},
  identifierstyle=\color{darkblue},
  keywordstyle=\color{cyan},
  morekeywords={xmlns,version,type}
}

\makeatletter
\newenvironment{CenteredBox}{%
\begin{Sbox}}{% Save the content in a box
\end{Sbox}\centerline{\parbox{\wd\@Sbox}{\TheSbox}}}% And output it centered
\makeatother



\title{Rechnersicherheit - Übung 11}
\author{Martin Görick, Dennis Hägler und Kai Kriedemann \\ Tutor: Stefan Pfeiffer}
%\date{\today}  % if the \date{} command is left out, the current date will be used


\begin{document}
\maketitle


\section*{Aufgabe 1 - RFC7525}
\begin{itemize}
  \item[1.] Zum Beispiel in der Email übertragung über POP3 und IMAP.
  \item[2.] SSLv2 und SSLv3 darf nicht verwendet werden. TLS 1.0 und 1.1 sollte nicht mehr verwendet werden. TLS 1.2 soll verwendet werden.
  \item[3.] HSTS dient dazu Man-in-the-middle-Angriffe zu verhindern. Dazu
\end{itemize}

\section*{Aufgabe 2 - SSL-Client}
\begin{itemize}
  \item[1.] Folgende Cipher Suites unterstützt OpenSSL version 1.0.1f unter
    Ubuntu:
    ECDHE-RSA-AES256-GCM-SHA384:ECDHE-ECDSA-AES256-GCM-SHA384:ECDHE-RSA-AES256-SHA384:ECDHE-ECDSA-AES256-SHA384:ECDHE-RSA-AES256-SHA:ECDHE-ECDSA-AES256-SHA:SRP-DSS-AES-256-CBC-SHA:SRP-RSA-AES-256-CBC-SHA:SRP-AES-256-CBC-SHA:DHE-DSS-AES256-GCM-SHA384:DHE-RSA-AES256-GCM-SHA384:DHE-RSA-AES256-SHA256:DHE-DSS-AES256-SHA256:DHE-RSA-AES256-SHA:DHE-DSS-AES256-SHA:DHE-RSA-CAMELLIA256-SHA:DHE-DSS-CAMELLIA256-SHA:ECDH-RSA-AES256-GCM-SHA384:ECDH-ECDSA-AES256-GCM-SHA384:ECDH-RSA-AES256-SHA384:ECDH-ECDSA-AES256-SHA384:ECDH-RSA-AES256-SHA:ECDH-ECDSA-AES256-SHA:AES256-GCM-SHA384:AES256-SHA256:AES256-SHA:CAMELLIA256-SHA:PSK-AES256-CBC-SHA:ECDHE-RSA-DES-CBC3-SHA:ECDHE-ECDSA-DES-CBC3-SHA:SRP-DSS-3DES-EDE-CBC-SHA:SRP-RSA-3DES-EDE-CBC-SHA:SRP-3DES-EDE-CBC-SHA:EDH-RSA-DES-CBC3-SHA:EDH-DSS-DES-CBC3-SHA:ECDH-RSA-DES-CBC3-SHA:ECDH-ECDSA-DES-CBC3-SHA:DES-CBC3-SHA:PSK-3DES-EDE-CBC-SHA:ECDHE-RSA-AES128-GCM-SHA256:ECDHE-ECDSA-AES128-GCM-SHA256:ECDHE-RSA-AES128-SHA256:ECDHE-ECDSA-AES128-SHA256:ECDHE-RSA-AES128-SHA:ECDHE-ECDSA-AES128-SHA:SRP-DSS-AES-128-CBC-SHA:SRP-RSA-AES-128-CBC-SHA:SRP-AES-128-CBC-SHA:DHE-DSS-AES128-GCM-SHA256:DHE-RSA-AES128-GCM-SHA256:DHE-RSA-AES128-SHA256:DHE-DSS-AES128-SHA256:DHE-RSA-AES128-SHA:DHE-DSS-AES128-SHA:DHE-RSA-SEED-SHA:DHE-DSS-SEED-SHA:DHE-RSA-CAMELLIA128-SHA:DHE-DSS-CAMELLIA128-SHA:ECDH-RSA-AES128-GCM-SHA256:ECDH-ECDSA-AES128-GCM-SHA256:ECDH-RSA-AES128-SHA256:ECDH-ECDSA-AES128-SHA256:ECDH-RSA-AES128-SHA:ECDH-ECDSA-AES128-SHA:AES128-GCM-SHA256:AES128-SHA256:AES128-SHA:SEED-SHA:CAMELLIA128-SHA:PSK-AES128-CBC-SHA:ECDHE-RSA-RC4-SHA:ECDHE-ECDSA-RC4-SHA:ECDH-RSA-RC4-SHA:ECDH-ECDSA-RC4-SHA:RC4-SHA:RC4-MD5:PSK-RC4-SHA:EDH-RSA-DES-CBC-SHA:EDH-DSS-DES-CBC-SHA:DES-CBC-SHA
  \item[2.] Der öffentlich Schlüssel von \textit{fu-berlin.de:443} ist 2048 Bit
    groß.
  \item[3.] Es wurde die Suite \textit{ECDHE-RSA-AES256-GCM-SHA384} verwendet.
    Dies bedeutet für den Schlüsselaustausch wird Elyptischen Kurven
    Diffie-Hellman verwendet der mit RSA signiert wird. Für das Verschlüssln
    wird AES mit 256 Bit im Counter Mode und der MAC Funktion SHA384 verwendet.
\end{itemize}

\section*{Aufgabe 3 - POODLE-Angriff}

\begin{itemize}
  \item[1.] Padding wird verwendet um die Länge des Klartextes anzupassen, wenn am Ende des Texten Bytes fehlen.
  \item[2.] In SSLv3 ist nicht spezifisiert, was der Padding enthält, sondern nur, das am Ende die Länge des Padding steht.
  \item[3.] SSLv3 wird noch verwendet, dies ist aber seid 15 Jahren veraltet und soll nicht verwendet werden.
  \item[4.] \textit{fu-berlin.de:443} verwendet TLSv1.2.
  \item[5.]
  \item[6.] Nein als minimum ist TSL 1.0 und maximum TSL 1.2 angegeben
\end{itemize}

\section*{Aufgabe 4 - Logjam-Angriff}

\begin{itemize}
  \item[1.] Es ist möglich, wenn ein Server den Cipher export unterstützt und kleiner als 2048 Bit Diffie-Hellman Gruppe.
  \item[2.] Für viele Verbindungen wurden die selbe generierte Schlüssel
    verwendet und da dieser nur aus der verwendeten Primzahl von Diffie-Hellman
    abhängig ist, welche auf vielen Servern gleich war, konnte dies ausgenutzt werden.
  \item[3.] Auf der Server Seite muss der Support für Cipher Suite export abgeschaltet werden und eine einzigartige ID von mindestens 2048 Bit Diffie-Hellman Gruppe erzeugen. Außerdem sollen TLS Bibliotheken min 2048 Bit Primzahllänge verwenden und alle Diffie-Hellman primzahlen die kleiner als 1024 Bit sind verwerfen.
  \item[4.] Es ist nicht möglich, da zum einen ein Schlüssel der Länge 2048 Bit verwendet wird. Außerdem kann mit dem folgendem Befehl getestet werden ob cipher export erlaubt ist.
    \begin{lstlisting}
      openssl s_client -connect fu-berlin.de:443 -cipher "EXP"
    \end{lstlisting}
    Aus Ausgabe erhält man \textit{Secure Renegotion is NOT supported}. Somit kann der FU Server nicht angegriffen werden.
\end{itemize}

\end{document}
