\documentclass{scrartcl}

\usepackage{graphicx} % Required for the inclusion of images

\renewcommand{\labelenumi}{\alph{enumi}.}
% deutsche Sprache und Umlaute
\usepackage[utf8x]{inputenc}
\usepackage[ngerman]{babel}


%BibTex verweise
\usepackage{cite}

% für url pfade
\usepackage{url}

% setzt beim neuen Absatz das erste Wort 4 mm nach rechts \\ kleine Abstand vom Rand
\setlength{\parindent}{4mm}

%für boxen
\usepackage{pifont,mdframed}

\usepackage{microtype}

% hack für scr paket für Indexierung
\usepackage{scrhack}

% erzwingen der Positionierung
\usepackage{float}
\restylefloat{figure}

% Verbindungen vom Index, Tabellen und Abbildungen
\usepackage{hyperref}

%Tiefe des Inhaltsverzeichnis
\setcounter{tocdepth}{5}

%für Unterstriche
\usepackage{underscore}

% Beschriftung
\usepackage{caption}
\captionsetup{%
  font=small,
  labelfont=bf,
}

\usepackage{fancybox}

%\geometry{showframe}% for debugging purposes -- displays the margins

\usepackage{amsmath}
\usepackage{amssymb}
\usepackage{geometry}
\geometry{verbose,a4paper,tmargin=20mm,bmargin=25mm,lmargin=15mm,rmargin=20mm}
% Set up the images/graphics package

% The following package makes prettier tables.  We're all about the bling!
\usepackage{booktabs}

% The units package provides nice, non-stacked fractions and better spacing
% for units.
\usepackage{units}

% The fancyvrb package lets us customize the formatting of verbatim
% environments.  We use a slightly smaller font.
\usepackage{fancyvrb}
\fvset{fontsize=\normalsize}

% Small sections of multiple columns
\usepackage{multicol}

% Provides paragraphs of dummy text
\usepackage{lipsum}

% listing
\usepackage{listings}
\usepackage{color}
\definecolor{grey}{rgb}{0.4,0.4,0.4}
\definecolor{darkblue}{rgb}{0.0,0.0,0.6}
\definecolor{cyan}{rgb}{0.0,0.6,0.6}

\lstset{
  basicstyle=\tiny,
  columns=fullflexible,
  showstringspaces=false,
  commentstyle=\color{gray}\upshape
}

\lstdefinelanguage{XML} {
  morestring=[b]",
  morestring=[s]{>}{<},
  morecomment=[s]{<?}{?>},
  stringstyle=\color{black},
  identifierstyle=\color{darkblue},
  keywordstyle=\color{cyan},
  morekeywords={xmlns,version,type}
}

\makeatletter
\newenvironment{CenteredBox}{%
\begin{Sbox}}{% Save the content in a box
\end{Sbox}\centerline{\parbox{\wd\@Sbox}{\TheSbox}}}% And output it centered
\makeatother



\title{Rechnersicherheit - Übung 07}
\author{Martin Görick, Dennis Hägler und Kai Kriedemann \\ Tutor: Stefan Pfeiffer}
%\date{\today}  % if the \date{} command is left out, the current date will be used


\begin{document}
\maketitle


\section*{Aufgabe 1 - HRU}
Folgende Lücke im Beweis: wie kommen wir von der Seite 68 mit den Behauptungen
\begin{itemize}
  \item $c_{1}$, ..., $c_{t}$ sind keine delete-Kommandos
  \item Höchstens $c_{1}$ kann ein create-Kommando sein
\end{itemize}

zu den auf Seite 69 genannten Behauptungen
\begin{itemize}
  \item $c_{1}$ ist create-Kommando und $c_{2}$ bis $c_{t}$ sind enter-Kommandos
  \item $c_{1}$, ..., $c_{t}$ sind enter-Kommandos
\end{itemize}

Dazu lassen sich bei Betrachtung der Basisoperationen folgende Beobachtungen vornehmen:

Wird ein create-Kommando durchgeführt, so erzeugt dies lediglich die Datei (ein Objekt). 
Es können jedoch keinerlei Rechte gesetzt werden. Mehrfaches Ausführen von create führt also 
zu Dateien ohne Rechte. Identifiziert werden diese Dateien jedoch durch ihre jeweiligen Rechte 
(in der Zugriffsmatrix). Durch das oben genannte Anlegen von Daten ohne Rechten sind quasi 
alle Daten äquivalent und können somit auf ein einziges create-Statement reduziert werden.

Mit den gegebenen Basisoperatoren kann man lediglich das Vorhandensein von Rechten prüfen. 
Es kann aber nicht umgekehrt geprüft werden, ob Rechte fehlen. Operationen wie "delete" 
haben keinerlei Auswirkungen, sofern r nicht existiert. Gleiches gilt für die Kommandos 
"destroy". Somit können diese für die Betrachtung ignoriert werden.

Somit kann man die auf Seite 68 genannten Aussagen auf die auf Seite 69 reduzieren.


\section*{Aufgabe 2 - Zugriffskontrolle(Linux)}
\subsection*{Welche Zugriffsrechte gibt es und was bedeuten diese?}
\begin{description}
  \item[Lesen / Read] \hfill \\
    Erlaubt lesenden Zugriff auf die Datei. Bei einem Verzeichnis können damit
    die Namen der enthaltenen Dateien und Ordner abgerufen werden (nicht jedoch
    deren weitere Daten wie z.B. Berechtigungen, Besitzer, Änderungszeitpunkt,
    Dateiinhalt etc.).
  \item[Schreiben / Write] \hfill \\
    Erlaubt schreibenden Zugriff auf eine Datei. Für ein Verzeichnis gesetzt,
    können Dateien oder Unterverzeichnisse angelegt oder gelöscht werden, sowie
    die Eigenschaften der enthaltenen Dateien/Verzeichnisse verändert werden.
  \item[Ausführen / Execute] \hfill \\
    Erlaubt das Ausführen einer Datei, wie das Starten eines Programms. Bei
    einem Verzeichnis ermöglicht dieses Recht, in diesen Ordner zu wechseln und
    weitere Attribute zu den enthaltenen Dateien abzurufen (sofern man die
    Dateinamen kennt ist dies unabhängig vom Leserecht auf diesen Ordner).
\end{description}

\subsection*{Abstraktion: Benutzergruppen, Protection Rings}
\subsubsection*{Benutzergruppen}
Jeder Benutzer ist einer Hauptgruppe zugeordnet, kann daneben aber auch
Mitglied weiterer Gruppen sein. Der Zugriff auf gewisse Hardware oder Dienste
ist auf die Mitglieder einer bestimmten Gruppe beschränkt.

Die Hauptgruppe eines Benutzers spielt im Zusammenhang mit den
Zugriffsberechtigungen auf Dateien und Verzeichnisse eine Rolle. Jede Datei ist
immer Eigentum genau eines Benutzers. Daneben ist den Dateien aber auch eine
Gruppe zugeordnet. Wenn ein Benutzer eine Datei erzeugt, dann wird seine
Hauptgruppe als Gruppe bei der erzeugten Datei eingetragen.


\paragraph{Namenskonventionen}
Die Namen von Benutzern und Gruppen unterliegen gewissen Einschränkungen:
\begin{itemize}
  \item Sie müssen mit einem Kleinbuchstaben beginnen.
  \item Danach können weitere Kleinbuchstaben, Ziffern (0-9), - (Minuszeichen)
    oder _ (Unterstrich) folgen. Zur Kompatibilität mit Konten auf
    Samba-Rechnern wird außerdem \$ am Ende des Benutzernamens unterstützt.
  \item Sie dürfen nur einmal auf dem System vergeben sein.
\end{itemize}
Diese Regeln sind in der Datei /etc/adduser.conf festgelegt.

\subsection*{Wer legt fur was welche Zugriffsrechte fest}
\subsection*{Wie wird die Zugriffsmatrix realisiert (Access Lists, Capabilities)}

\begin{itemize}
  \item In Linux (sowie in Windows) realisiert durch Zugriffslisten (ACL). Unterschieden wird 
  dabei zwischen Access ACL (Zugriffsrechte für Gruppen und Benutzer auf beliebige Dateisystemobjekte), 
  Default ACL (Anwendung nur auf Verzeichnisse; Festlegung, welche Rechte vom übergeordneten
  Verzeichnis geerbt werden können) .
  \item Ein ACL Eintrag hat dabei im Grunde zwei verschiedene Formen – minimal und erweitert. Ein 
  minimaler ACL-Eintrag besteht aus dem Typ owner, owning group und other. Eine erweiterte enthält 
  noch einen sogenannten mask-Eintrag und darf damit mehrere Einträge des Typs named user und named 
  group enthalten.
  \item Quelle: \url{http://users.suse.com/~agruen/acl/chapter/fs_acl-de.pdf}
\end{itemize}

\subsection*{Wie wird die Zugriffsmatrix geschutzt (Reference Monitor)}

\begin{itemize}
  \item LSM ist ein Framework und eine Schnittstelle für einen reference monitor. Eine generelle 
  Implementierung dieser Schnittstelle erfolgt im sogenannten SELinux. Generelles grobes Vorgehen: 
  Aufruf von LSM hooks via Systemaufrufe; diese hooks besitzen Referenzen auf Linux Objekte 
  Reference Monitor muss diese jedoch in konforme Authorisierungsqueries
  transformieren; desweiteren müssen dieses Authorisierungsqueries für das Protection System 
  verarbeitet werden
  \item Quelle: \url{http://www.cse.psu.edu/~trj1/cse443-s12/docs/ch9.pdf}
\end{itemize}

\end{document}
