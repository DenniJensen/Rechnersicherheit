\documentclass{scrartcl}

\usepackage{graphicx} % Required for the inclusion of images

\renewcommand{\labelenumi}{\alph{enumi}.}
% deutsche Sprache und Umlaute
\usepackage[utf8x]{inputenc} 
\usepackage[ngerman]{babel}


%BibTex verweise
\usepackage{cite}

% für url pfade
\usepackage{url}

% setzt beim neuen Absatz das erste Wort 4 mm nach rechts \\ kleine Abstand vom Rand
\setlength{\parindent}{4mm}

%für boxen
\usepackage{pifont,mdframed}

\usepackage{microtype}

% hack für scr paket für Indexierung
\usepackage{scrhack}

% erzwingen der Positionierung
\usepackage{float}
\restylefloat{figure}

% Verbindungen vom Index, Tabellen und Abbildungen
\usepackage{hyperref}

%Tiefe des Inhaltsverzeichnis
\setcounter{tocdepth}{5}

%für Unterstriche
\usepackage{underscore} 

% Beschriftung
\usepackage{caption}
\captionsetup{%
  font=small,
  labelfont=bf,
}

\usepackage{fancybox}

%\geometry{showframe}% for debugging purposes -- displays the margins

\usepackage{amsmath}
\usepackage{geometry}
\geometry{verbose,a4paper,tmargin=20mm,bmargin=25mm,lmargin=15mm,rmargin=20mm}
% Set up the images/graphics package

% The following package makes prettier tables.  We're all about the bling!
\usepackage{booktabs}

% The units package provides nice, non-stacked fractions and better spacing
% for units.
\usepackage{units}

% The fancyvrb package lets us customize the formatting of verbatim
% environments.  We use a slightly smaller font.
\usepackage{fancyvrb}
\fvset{fontsize=\normalsize}

% Small sections of multiple columns
\usepackage{multicol}

% Provides paragraphs of dummy text
\usepackage{lipsum}

% listing
\usepackage{listings}
\usepackage{color}
\definecolor{grey}{rgb}{0.4,0.4,0.4}
\definecolor{darkblue}{rgb}{0.0,0.0,0.6}
\definecolor{cyan}{rgb}{0.0,0.6,0.6}

\lstset{
  basicstyle=\tiny,
  columns=fullflexible,
  showstringspaces=false,
  commentstyle=\color{gray}\upshape
}

\lstdefinelanguage{XML} {
  morestring=[b]",
  morestring=[s]{>}{<},
  morecomment=[s]{<?}{?>},
  stringstyle=\color{black},
  identifierstyle=\color{darkblue},
  keywordstyle=\color{cyan},
  morekeywords={xmlns,version,type}
}

\makeatletter
\newenvironment{CenteredBox}{% 
\begin{Sbox}}{% Save the content in a box
\end{Sbox}\centerline{\parbox{\wd\@Sbox}{\TheSbox}}}% And output it centered
\makeatother

 

\title{Rechnersicherheit - Übung 04}
\author{Dennis Hägler und Martin Görick}
%\date{\today}  % if the \date{} command is left out, the current date will be used


\begin{document}
\maketitle


\section*{Aufgabe 1}
Zuerst wird mittels einer Blockchiffre aus dem geheimen Schlüssel $K$ zwei Schlüssel $K_1$ und $K_2$ generiert.
Danach wird die MAC nach Abbildung \ref{fig:cmac} generiert.

\textbf{CHIP} ist die Blockchiffre mit dem geheimen Schlüssel K
\textbf{$MBS_Tlen$} ist die Länge der MAC

\begin{figure}[thp]
  \begin{CenteredBox}
\includegraphics[scale=0.5]{CMAC}
  \end{CenteredBox}
  \caption{MAC generierung}
  \label{fig:cmac}
\end{figure}

Zur Verifikation wird aus dem $K,M$ ein $T'$ berechnet und mit dem übermittelten $T$ verglichen.

\paragraph*{Quelle}
Recommendation for Block Cipher Modes of Operation: The CMAC Mode for Authentication, Morris Dworkin, 	NIST Special Publication 800-38B, May 2005. 


\section*{Aufgabe 2}
\begin{itemize}
\item[1] Ein Empfänger welcher den Schlüssel zum vergleichen hat, könnte mit einer zufälligen Signatur, zufällige aussehende Texte erstellen. Damit kann verhindert werden, dass der Sender zufällige Texte wie Schlüssel signieren kann, da nicht sicher ist ob die zufälligen Texte vom Sender sind. 
\item[2] Ein Empfänger will aus einem Text m und einer Signatur s ein eigenes Dokument $m''$ mit der gültigen Signatur $s''$ erzeugen. Unter einem Vorwand soll der Sender ein Dokument $m' = m^{-1} * m''$ signieren. Der Sender soll keinen Verdacht schöpfen, da $m'$ unsinnig ist. Mit der erhaltenen Signatur $s'$ unter mod n gerechnet und mit dem privaten Schlüssel $d$ erhält der Angreifer ein $s''$ wie folgt.
$$s'' = s * s' = m^d * m'^d = (m*m')^d = (m*m6{-1} * m'')^d = m''^d$$
\end{itemize}

\paragraph*{Quelle}
RSA-Signatur,Hans Werner Lang, 2014, \url{http://www.iti.fh-flensburg.de/lang/krypto/protokolle/rsa-signatur.htm}

\section*{Aufgabe 3}

\begin{itemize}
\item[a)] 
\item[b)]
\end{itemize}

\section*{Aufgabe 4}

\end{document}
