\documentclass{scrartcl}

\usepackage{graphicx} % Required for the inclusion of images

\renewcommand{\labelenumi}{\alph{enumi}.}
% deutsche Sprache und Umlaute
\usepackage[utf8x]{inputenc}
\usepackage[ngerman]{babel}


%BibTex verweise
\usepackage{cite}

% für url pfade
\usepackage{url}

% setzt beim neuen Absatz das erste Wort 4 mm nach rechts \\ kleine Abstand vom Rand
\setlength{\parindent}{4mm}

%für boxen
\usepackage{pifont,mdframed}

\usepackage{microtype}

% hack für scr paket für Indexierung
\usepackage{scrhack}

% erzwingen der Positionierung
\usepackage{float}
\restylefloat{figure}

% Verbindungen vom Index, Tabellen und Abbildungen
\usepackage{hyperref}

%Tiefe des Inhaltsverzeichnis
\setcounter{tocdepth}{5}

%für Unterstriche
\usepackage{underscore}

% Beschriftung
\usepackage{caption}
\captionsetup{%
  font=small,
  labelfont=bf,
}

\usepackage{fancybox}

%\geometry{showframe}% for debugging purposes -- displays the margins

\usepackage{amsmath}
\usepackage{amssymb}
\usepackage{geometry}
\geometry{verbose,a4paper,tmargin=20mm,bmargin=25mm,lmargin=15mm,rmargin=20mm}
% Set up the images/graphics package

% The following package makes prettier tables.  We're all about the bling!
\usepackage{booktabs}

% The units package provides nice, non-stacked fractions and better spacing
% for units.
\usepackage{units}

% The fancyvrb package lets us customize the formatting of verbatim
% environments.  We use a slightly smaller font.
\usepackage{fancyvrb}
\fvset{fontsize=\normalsize}

% Small sections of multiple columns
\usepackage{multicol}

% Provides paragraphs of dummy text
\usepackage{lipsum}

% listing
\usepackage{listings}
\usepackage{color}
\definecolor{grey}{rgb}{0.4,0.4,0.4}
\definecolor{darkblue}{rgb}{0.0,0.0,0.6}
\definecolor{cyan}{rgb}{0.0,0.6,0.6}

\lstset{
  basicstyle=\tiny,
  columns=fullflexible,
  showstringspaces=false,
  commentstyle=\color{gray}\upshape
}

\lstdefinelanguage{XML} {
  morestring=[b]",
  morestring=[s]{>}{<},
  morecomment=[s]{<?}{?>},
  stringstyle=\color{black},
  identifierstyle=\color{darkblue},
  keywordstyle=\color{cyan},
  morekeywords={xmlns,version,type}
}

\makeatletter
\newenvironment{CenteredBox}{%
\begin{Sbox}}{% Save the content in a box
\end{Sbox}\centerline{\parbox{\wd\@Sbox}{\TheSbox}}}% And output it centered
\makeatother



\title{Rechnersicherheit - Übung 04}
\author{Dennis Hägler und Martin Görick \\ Tutor: Stefan Pfeiffer}
%\date{\today}  % if the \date{} command is left out, the current date will be used


\begin{document}
\maketitle


\section{Aufgabe 1 - }
\section{Aufgabe 2 - qualifizierte elektronische Signatur}
Das deutsche Signaturgesetz (SigG) unterscheidet elektronische Signaturen,
fortgeschrittene elektronische Signaturen und qualifizierte elektronische
Signaturen. Qulifizierte elektronische Signaturen sind eigentlich Inhalt des
Gesetzes.

\subsubsection{Signaturgesetetz}
\begin{itemize}
\itme private Keys muessen auf Hardware gespeichert werden, aus der sie nicht
gelesen werden koennen (z.B. Smartcars)
\end{itemize}

\section{Aufgabe 3 - OpenPGP Protol}
\subsection{Signatur}
\begin{itemize}
  \item OpenPGP unterstuetz digitale Signaturen
  \item Absender bildet Hash-Wert und signiert dieses
  \item Empfaenger prueft Signatur mit public key
\end{itemize}

\subsection{Schluesselbeglaubigung}
\begin{itemize}
  \item basiert auf Web of Trust (jeder Teilnehmer kann Schluessel des anderen
    signieren)
  \item Name, E-Mail und Kommentar ermoeglichen Dritten die Authentiziteat des
    Zertifikates einzuschaetzen
  \item Zertifizierung bezieht sich nur auf die Echtheit des Schluessels
  \item Die Gueltigkeit von Zertifikation ist eine oeffentliche Informaiton
  \item Echtheit kann auch mit Fingerabdruck (Pruefsumme verglichen werden)
\end{itemize}

\end{document}
