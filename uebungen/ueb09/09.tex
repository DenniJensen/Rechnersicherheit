\documentclass{scrartcl}

\usepackage{graphicx} % Required for the inclusion of images

\renewcommand{\labelenumi}{\alph{enumi}.}
% deutsche Sprache und Umlaute
\usepackage[utf8x]{inputenc}
\usepackage[ngerman]{babel}


%BibTex verweise
\usepackage{cite}

% für url pfade
\usepackage[hyphens]{url}

% setzt beim neuen Absatz das erste Wort 4 mm nach rechts \\ kleine Abstand vom Rand
\setlength{\parindent}{4mm}

%für boxen
\usepackage{pifont,mdframed}

\usepackage{microtype}

% hack für scr paket für Indexierung
\usepackage{scrhack}

% erzwingen der Positionierung
\usepackage{float}
\restylefloat{figure}

% Verbindungen vom Index, Tabellen und Abbildungen
\usepackage{hyperref}

%Tiefe des Inhaltsverzeichnis
\setcounter{tocdepth}{5}

%für Unterstriche
\usepackage{underscore}

% Beschriftung
\usepackage{caption}
\captionsetup{%
  font=small,
  labelfont=bf,
}

\usepackage{fancybox}

%\geometry{showframe}% for debugging purposes -- displays the margins

\usepackage{amsmath}
\usepackage{amssymb}
\usepackage{geometry}
\geometry{verbose,a4paper,tmargin=20mm,bmargin=25mm,lmargin=15mm,rmargin=20mm}
% Set up the images/graphics package

% The following package makes prettier tables.  We're all about the bling!
\usepackage{booktabs}

% The units package provides nice, non-stacked fractions and better spacing
% for units.
\usepackage{units}

% The fancyvrb package lets us customize the formatting of verbatim
% environments.  We use a slightly smaller font.
\usepackage{fancyvrb}
\fvset{fontsize=\normalsize}

% Small sections of multiple columns
\usepackage{multicol}

% Provides paragraphs of dummy text
\usepackage{lipsum}

% listing
\usepackage{listings}
\usepackage{color}
\definecolor{grey}{rgb}{0.4,0.4,0.4}
\definecolor{darkblue}{rgb}{0.0,0.0,0.6}
\definecolor{cyan}{rgb}{0.0,0.6,0.6}

\lstset{
  basicstyle=\tiny,
  columns=fullflexible,
  showstringspaces=false,
  commentstyle=\color{gray}\upshape
}

\lstdefinelanguage{XML} {
  morestring=[b]",
  morestring=[s]{>}{<},
  morecomment=[s]{<?}{?>},
  stringstyle=\color{black},
  identifierstyle=\color{darkblue},
  keywordstyle=\color{cyan},
  morekeywords={xmlns,version,type}
}

\makeatletter
\newenvironment{CenteredBox}{%
\begin{Sbox}}{% Save the content in a box
\end{Sbox}\centerline{\parbox{\wd\@Sbox}{\TheSbox}}}% And output it centered
\makeatother



\title{Rechnersicherheit - Übung 09}
\author{Martin Görick, Dennis Hägler und Kai Kriedemann \\ Tutor: Stefan Pfeiffer}
%\date{\today}  % if the \date{} command is left out, the current date will be used


\begin{document}
\maketitle


\section*{Aufgabe 1 - Bell-LaPadula}
Damit die formulierten Ziele erfüllt werden, müssen ii und iii um folgendes erweitert werden.

\begin{itemize}
\item[ii] $sc(o) \leq sc(s)$, Die Sicherheitsklasse des Objektes muss kleiner oder gleich der Sicherheitsklasse des Subjektes sein, damit \textit{no read up} in den Sicherheitsklassen gewährleistet ist.
\item[iii] $sc(s) \leq sc(o)$, Damit das \textit{no write down} Prinzip gewährleistet ist, muss die Sicherheitsklasse des Objekts gleich oder größer als die des Subjekts sein.
\end{itemize}

\section*{Aufgabe 2 - Biba-Integrity-Model}

Das Bell-LaPadula Modell wird um Integrität Klassen erweitert. Diese sind ähnlich der Sicherheitsklassen, aber unabhängig von diesen.

\begin{itemize}
\item \textbf{kritisch} - wenn die Informationen von nicht authentisieren Personen modifiziert werden, können diese die Nationale Sicherheit \underline{sehr schwer beschädigen}.
\item \textbf{sehr wichtig} - wenn die Informationen von nicht authentisieren Personen modifiziert werden, können diese die Nationale Sicherheit \underline{ernst beschädigen}.
\item \textbf{wichtig} - wenn die Informationen von nicht authentisieren Personen modifiziert werden, können diese die Nationale Sicherheit \underline{beschädigen}
\end{itemize}

Dazu werden die Methoden zum festlegen der Integrität ($il(s)$),vergleichen, ob die Integritäten kleiner gleich sind oder ob ein Subjekt ein anderen Subjekt aufrufen darf.

\begin{table}[H]
\begin{center}
\begin{tabular}{c|c|c|c}
 & $sl(s) \leq sl(o)$ & $cl(s) = cl(o)$ & $cl(o) \leq cl(s)$ \\
 \hline
 $il(s) \leq il(o)$ & & read & read \\
 \hline
 $il(s) = il(o)$ & write &  read,write & read \\
\hline
$il(o) \leq il(s)$ & write & write & \\
\end{tabular}
\end{center}
\caption{Sicherheit und Integrität Einschränkungen}
\label{tab:2}
\end{table}

Wie in der Tabelle \ref{tab:2} zu sehen, wird im Bell-LaPaluda Modell statt nur die Sicherheitsklassen zu testen auch die Intengritätsklassen verglichen.

\section*{Aufgabe 3 - Flickercode}

\begin{itemize}
\item \texttt{Phishing:} Nicht möglich, da der Angreifer den genauen Flickercode der \textit{falschen} Überweisung benötigt, welche die Bank generiert um damit die TAN für diese zu erhalten.  

Ist zwar immer noch möglich (falsche Website, kriegt Details der Transaktion durch Übertragung des Flickercodes);
allerdings kann/muss der Kunde vor dem Abschicken die Details noch einmal überprüfen; dort sollte eine Manipulation
sofort auffallen (man sieht Summe, Ziel-Kontonr.,etc.); Schwierigkeiten machen mehrere Überweisungen bei Verwendung
einer TAN; dort ist es nicht möglich, Details wie Empfänger und co, sondern lediglich die Gesamtsumme zu sehen; somit
wären hier Manipulationen möglich; eine weitere Möglichkeit ist es, eine einzelene Überweisung in eine Gruppenüberwei
sung von 1 umzuwandeln; dann wird bei der finalen Bestätigung dort ebenfalls die Ziel-Kontonr. nicht angezeigt; auch
dies sollte dem Bestätiger auffallen; beide Fälle wurden aber bereits von den Banken behoben

\item \texttt{Key-Logging:} Da die TAN nur für eine Überweisung gilt bringt ihm das Logging der TAN nicht, jedoch wenn
er es schafft einen Keylogger zu installieren, dann eventuell auch einen Trojaner und damit wurde dieses Verfahren
durch falsche Anzeigen am Bildschirm, Anwender rein gelegt. (Quelle: \url{http://www.heise.de/security/meldung/Onlin
e-Banking-Trojaner-hat-es-auf-chipTAN-Nutzer-abgesehen-1701184.html})

Loggt der Angreifer die Details der Transaktion mit (falls Daten am PC und nicht am Gerät eingegeben werden), fehlt ihm jedoch der Flickercode bzw. die optische Übertragung der Signale; somit hat er keinerlei Chancen

\item \texttt{Replay:} Geht nicht, da die TAN nur einmal gültig ist.

Hier schneidet ein Angreifer einen vergangenen Flickercode (Startcode, Kontonr. Empfänger, Überweisungsbetrag;
verschlüsselt in einer Grafik) mit. Selbst, wenn dies gelingen würde (hypothetisch: Bank hat keinerlei Maßnahmen gegen
alte Transaktionen), muss der TAN-Generator des originalen Benutzers eine TAN generieren und manuell freigeben. Bank
überprüft dann, ob der TAN passt. Da der Angreifer das Gerät aber nicht hat bzw. die passende Karte nicht, gelingt
dies nicht, da er keine TAN hat. 
Schneidet der Angreifer nun eine TAN mit, so hat er auch hier keinerlei Chancen, da die TANs Einmal-TANs sind.


\end{itemize}

\end{document}
