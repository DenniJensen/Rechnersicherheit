\documentclass{scrartcl}

\usepackage{graphicx} % Required for the inclusion of images

\renewcommand{\labelenumi}{\alph{enumi}.}
% deutsche Sprache und Umlaute
\usepackage[utf8x]{inputenc}
\usepackage[ngerman]{babel}


%BibTex verweise
\usepackage{cite}

% für url pfade
\usepackage{url}

% setzt beim neuen Absatz das erste Wort 4 mm nach rechts \\ kleine Abstand vom Rand
\setlength{\parindent}{4mm}

%für boxen
\usepackage{pifont,mdframed}

\usepackage{microtype}

% hack für scr paket für Indexierung
\usepackage{scrhack}

% erzwingen der Positionierung
\usepackage{float}
\restylefloat{figure}

% Verbindungen vom Index, Tabellen und Abbildungen
\usepackage{hyperref}

%Tiefe des Inhaltsverzeichnis
\setcounter{tocdepth}{5}

%für Unterstriche
\usepackage{underscore}

% Beschriftung
\usepackage{caption}
\captionsetup{%
  font=small,
  labelfont=bf,
}

\usepackage{fancybox}

%\geometry{showframe}% for debugging purposes -- displays the margins

\usepackage{amsmath}
\usepackage{amssymb}
\usepackage{geometry}
\geometry{verbose,a4paper,tmargin=20mm,bmargin=25mm,lmargin=15mm,rmargin=20mm}
% Set up the images/graphics package

% The following package makes prettier tables.  We're all about the bling!
\usepackage{booktabs}

% The units package provides nice, non-stacked fractions and better spacing
% for units.
\usepackage{units}

% The fancyvrb package lets us customize the formatting of verbatim
% environments.  We use a slightly smaller font.
\usepackage{fancyvrb}
\fvset{fontsize=\normalsize}

% Small sections of multiple columns
\usepackage{multicol}

% Provides paragraphs of dummy text
\usepackage{lipsum}

% listing
\usepackage{listings}
\usepackage{color}
\definecolor{grey}{rgb}{0.4,0.4,0.4}
\definecolor{darkblue}{rgb}{0.0,0.0,0.6}
\definecolor{cyan}{rgb}{0.0,0.6,0.6}

\lstset{
  basicstyle=\tiny,
  columns=fullflexible,
  showstringspaces=false,
  commentstyle=\color{gray}\upshape
}

\lstdefinelanguage{XML} {
  morestring=[b]",
  morestring=[s]{>}{<},
  morecomment=[s]{<?}{?>},
  stringstyle=\color{black},
  identifierstyle=\color{darkblue},
  keywordstyle=\color{cyan},
  morekeywords={xmlns,version,type}
}

\makeatletter
\newenvironment{CenteredBox}{%
\begin{Sbox}}{% Save the content in a box
\end{Sbox}\centerline{\parbox{\wd\@Sbox}{\TheSbox}}}% And output it centered
\makeatother



\title{Rechnersicherheit - Übung 06}
\author{Dennis Hägler und Martin Görick \\ Tutor: Stefan Pfeiffer}
%\date{\today}  % if the \date{} command is left out, the current date will be used


\begin{document}
\maketitle


\section*{Aufgabe 1 - Diffie-Hellman}
\subsection{Station-to-Station-Protokoll}
\begin{itemize}
  \item verwendung von difitalen Signaturen(RSA) und MAC
\end{itemize}

\section*{Aufgabe 2 - SPEKE}

\section*{Aufgabe 3 - Needham-Schroeder}
\begin{itemize}
  \item jeweils geheimer Schlüssel zwischen Teilnehmer und Authentication Server
  \item durch Authentifizierung der Teilnehmer mit AS, sind Teilnehmer
    untereinander authentisiert
\end{itemize}
\begin{enumerate}
  \item Teilnehmer A versendet verschl. Nachricht an AS (Authentizität und
    gewünschter Korrespondenten enthalten, sowie eine Nonce)
  \item AS nutzt geheimen Schlüssel beider Kommunikationspartner (Antwort an A
    verschlüsselt mit dem Schlüssel von A, enthält Nonce(Sicherstellung ob
    nicht Aufgezeichnete Nachricht)). Nachricht enthält Identität von B.
    Letzte Block ist mit dem Schlüssel von B verschlüsselt(befindet sich in der
    verschlüsselten Nachricht an A)
  \item A sendet an B (verschlüsselt mit B Schlüssel). B kann Nachricht
    entschlüsseln und weiß dass A eine sichere Verbindung mit dem
    Sitzungschlüssel durchführen möchte. A und B besitzen in dem Moment den
    Sitzungschlüssel. A weiß dass jede Nachricht, die mit Sitzungschlüssel
    verschlüsselt wurde von B stammt.
\end{enumerate}

\paragraph*{Asymmetrisch}
\begin{itemize}
  \item A will mit B kommunizieren
  \item A und B brauchen jeweils die Public Keys des jeweils anderen
  \item Teilinstanz T wird dafür benötigt
  \item T kennt A und B
  \item beugt Man-In-The-Middle-Angriff vor
\end{itemize}

\begin{enumerate}
  \item A sendet an T unverschlüsselte Nachricht (will mit B sprechen)
  \item T antwortet mit public Key von B + Identität von B (mit private von T
    signiert) (stellt sicher dass die Antwort von T kommt)
  \item A kann mit public Key von B an B senden (verschlüsselt mit Nonce NA und
    A als Absender)
\end{enumerate}

\subsection*{Angriff von Lowe}
\paragraph*{Initial zustand}
\begin{itemize}
\item Key Server kennt Public key aller Beteiligten auch Angreifer O
\item Alice kennt O und vertraut ihm
\item O kennt nur eigenes Schlüsselpaar $K_O$
\end{itemize}

\paragraph*{Szenario}
\begin{itemize}
\item Alice möchte mit O reden
\item O will mit Hilfe von Alice Authentisierungsnachrichten gegenüber Bob B als Alice ausgeben O(A)
\end{itemize}

\paragraph*{Lowe attack}
\begin{itemize}
\item[1.] $A \rightarrow O: \lbrace N_A, A \rbrace_{K_O}$
Nonce von A, ID A, verschlüsselt mit PK von O
\item[2.] $O(A) \rightarrow B: \lbrace N_A, A \rbrace_{K_B} $
O verschlüsselt Nachricht von A mit PK von B um sich als A auszugeben
\item[3.] $B \rightarrow O(A): \lbrace N_A, N_B \rbrace_{K_A}$
B denkt, dass er mit A kommuniziert und verschlüsselt seine Nachricht mit PK von A
(O kann Nachricht nicht lesen)
\item[4.] $O \rightarrow A: \lbrace N_A, N_B \rbrace_{K_A}$
O leitet die Nachricht unverändert weiter.
\item[5.] $ A \rightarrow O: \lbrace N_B \rbrace_{K_O}$
A sendet Nonce von B zu O, da sie denkt, dass $N_B$ von O kommt und will sich damit authentisieren
\item[6.] $ O(A) \rightarrow B: \lbrace N_B \rbrace_{K_B}$
O sendet $N_B$ neu verschlüsselt an B. Nun denkt B das er mit A kommuniziert, da sie in der Lage ist die Nonce zu entschlüsseln
\end{itemize}

\subparagraph*{Lösung}
Die Nachricht 3 enthält nicht nur $N_A$ und $N_B$ sondern auch die ID von B.
Damit erkennt A, dass sie nicht mit O redet und bricht die Verbindung ab.

\end{document}
