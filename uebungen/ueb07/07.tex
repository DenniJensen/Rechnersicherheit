\documentclass{scrartcl}

\usepackage{graphicx} % Required for the inclusion of images

\renewcommand{\labelenumi}{\alph{enumi}.}
% deutsche Sprache und Umlaute
\usepackage[utf8x]{inputenc}
\usepackage[ngerman]{babel}


%BibTex verweise
\usepackage{cite}

% für url pfade
\usepackage{url}

% setzt beim neuen Absatz das erste Wort 4 mm nach rechts \\ kleine Abstand vom Rand
\setlength{\parindent}{4mm}

%für boxen
\usepackage{pifont,mdframed}

\usepackage{microtype}

% hack für scr paket für Indexierung
\usepackage{scrhack}

% erzwingen der Positionierung
\usepackage{float}
\restylefloat{figure}

% Verbindungen vom Index, Tabellen und Abbildungen
\usepackage{hyperref}

%Tiefe des Inhaltsverzeichnis
\setcounter{tocdepth}{5}

%für Unterstriche
\usepackage{underscore}

% Beschriftung
\usepackage{caption}
\captionsetup{%
  font=small,
  labelfont=bf,
}

\usepackage{fancybox}

%\geometry{showframe}% for debugging purposes -- displays the margins

\usepackage{amsmath}
\usepackage{amssymb}
\usepackage{geometry}
\geometry{verbose,a4paper,tmargin=20mm,bmargin=25mm,lmargin=15mm,rmargin=20mm}
% Set up the images/graphics package

% The following package makes prettier tables.  We're all about the bling!
\usepackage{booktabs}

% The units package provides nice, non-stacked fractions and better spacing
% for units.
\usepackage{units}

% The fancyvrb package lets us customize the formatting of verbatim
% environments.  We use a slightly smaller font.
\usepackage{fancyvrb}
\fvset{fontsize=\normalsize}

% Small sections of multiple columns
\usepackage{multicol}

% Provides paragraphs of dummy text
\usepackage{lipsum}

% listing
\usepackage{listings}
\usepackage{color}
\definecolor{grey}{rgb}{0.4,0.4,0.4}
\definecolor{darkblue}{rgb}{0.0,0.0,0.6}
\definecolor{cyan}{rgb}{0.0,0.6,0.6}

\lstset{
  basicstyle=\tiny,
  columns=fullflexible,
  showstringspaces=false,
  commentstyle=\color{gray}\upshape
}

\lstdefinelanguage{XML} {
  morestring=[b]",
  morestring=[s]{>}{<},
  morecomment=[s]{<?}{?>},
  stringstyle=\color{black},
  identifierstyle=\color{darkblue},
  keywordstyle=\color{cyan},
  morekeywords={xmlns,version,type}
}

\makeatletter
\newenvironment{CenteredBox}{%
\begin{Sbox}}{% Save the content in a box
\end{Sbox}\centerline{\parbox{\wd\@Sbox}{\TheSbox}}}% And output it centered
\makeatother



\title{Rechnersicherheit - Übung 07}
\author{Martin Görick Kai Kriedemann und Dennis Hägler \\ Tutor: Stefan Pfeiffer}
%\date{\today}  % if the \date{} command is left out, the current date will be used


\begin{document}
\maketitle


\section*{Aufgabe 1 - Seed-Generierung}
\begin{itemize}
  \item von der Theorie abhänging von einander
  \item praktisch nicht, da exakte Taktfrequenz nicht simuliert werden kann
  \item Systemstart und Prozessortakt ist unterschiedlich
  \item dazu kommen unterschiede durch Netzwerkabhängigkeiten
\end{itemize}

\section*{Aufgabe 2 - Deterministische Zufallszahlengeneratore}
\subsection*{kryptographisch}
\begin{itemize}
  \item Hashfunktion basiert auf Einwegfunktion
  \item Seed ist zufällig und kann nicht reproduziert werden
  \item sollte Hashfunktion rückgängig berechnet werden können, kann nicht
    bestimmt werden ob Seed richtig ist, da er zufällig entstanden ist
\end{itemize}

\subsection*{Blockchiffren im Counter-Mode}
\begin{itemize}
  \item keine Abhängigkeit zu Vor- oder Nachfolger Block
  \item Komplett parallel durchführbar
  \item durch den Counter verändert sich der Input, daraus entsteht komplett
    verschiedener output
\end{itemize}

\section*{Aufgabe 3 - Instanzauthentisierung}
\subsection*{Passwordlisten}
\begin{itemize}
  \item schutz vor Key-Loggin und Replay Attacken, da Passowrd nur einmalig
    benutzt werden können und dannach verfallen
  \item gegen Phiishing nicht sicher da Passwörter abgefangen und selbst
    benutzt werden können (Passwörter wurden im Echtsystem nicht genutzt und
    sind dadurch noch aktiv)
\end{itemize}

\subsection*{Passwordgenerator}
\begin{itemize}
  \item im gegebenen Zeitintervall nicht vor Replay Attacken sicher (Passwort
    verfällt erst nach dem Zeitintervall und ist somit noch gültig)
  \item Phishing ist innerhalb des Zeitintervalls möglich
  \item Key-Loggin ist nur innerhalb des Zeitintervalls möglich wenn das
    Password sofort weiter übermittelt wird und genutzt wird
\end{itemize}

\section*{Aufgabe 4 - Passwörter}

\end{document}
