\documentclass{scrartcl}

\usepackage{graphicx} % Required for the inclusion of images

\renewcommand{\labelenumi}{\alph{enumi}.}
% deutsche Sprache und Umlaute
\usepackage[utf8x]{inputenc}
\usepackage[ngerman]{babel}


%BibTex verweise
\usepackage{cite}

% für url pfade
\usepackage[hyphens]{url}

% setzt beim neuen Absatz das erste Wort 4 mm nach rechts \\ kleine Abstand vom Rand
\setlength{\parindent}{4mm}

%für boxen
\usepackage{pifont,mdframed}

\usepackage{microtype}

% hack für scr paket für Indexierung
\usepackage{scrhack}

% erzwingen der Positionierung
\usepackage{float}
\restylefloat{figure}

% Verbindungen vom Index, Tabellen und Abbildungen
\usepackage{hyperref}

%Tiefe des Inhaltsverzeichnis
\setcounter{tocdepth}{5}

%für Unterstriche
\usepackage{underscore}

% Beschriftung
\usepackage{caption}
\captionsetup{%
  font=small,
  labelfont=bf,
}

\usepackage{fancybox}

%\geometry{showframe}% for debugging purposes -- displays the margins

\usepackage{amsmath}
\usepackage{amssymb}
\usepackage{geometry}
\geometry{verbose,a4paper,tmargin=20mm,bmargin=25mm,lmargin=15mm,rmargin=20mm}
% Set up the images/graphics package

% The following package makes prettier tables.  We're all about the bling!
\usepackage{booktabs}

% The units package provides nice, non-stacked fractions and better spacing
% for units.
\usepackage{units}

% The fancyvrb package lets us customize the formatting of verbatim
% environments.  We use a slightly smaller font.
\usepackage{fancyvrb}
\fvset{fontsize=\normalsize}

% Small sections of multiple columns
\usepackage{multicol}

% Provides paragraphs of dummy text
\usepackage{lipsum}

% listing
\usepackage{listings}
\usepackage{color}
\definecolor{grey}{rgb}{0.4,0.4,0.4}
\definecolor{darkblue}{rgb}{0.0,0.0,0.6}
\definecolor{cyan}{rgb}{0.0,0.6,0.6}

\lstset{
  basicstyle=\tiny,
  columns=fullflexible,
  showstringspaces=false,
  commentstyle=\color{gray}\upshape
}

\lstdefinelanguage{XML} {
  morestring=[b]",
  morestring=[s]{>}{<},
  morecomment=[s]{<?}{?>},
  stringstyle=\color{black},
  identifierstyle=\color{darkblue},
  keywordstyle=\color{cyan},
  morekeywords={xmlns,version,type}
}

\makeatletter
\newenvironment{CenteredBox}{%
\begin{Sbox}}{% Save the content in a box
\end{Sbox}\centerline{\parbox{\wd\@Sbox}{\TheSbox}}}% And output it centered
\makeatother



\title{Rechnersicherheit - Übung 11}
\author{Martin Görick, Dennis Hägler und Kai Kriedemann \\ Tutor: Stefan Pfeiffer}
%\date{\today}  % if the \date{} command is left out, the current date will be used


\begin{document}
\maketitle


\section*{Aufgabe 1 - Rainbow Tables}


\section*{Aufgabe 2 - Android-Sicherheit}



\section*{Aufgabe 3 - Wie kann Overflow verhindert werden}
\subsubsection{Programmierer}
  \begin{itemize}
    \item genug Speicher reservieren
    \item sorgfälltig mit der Pointern umgehen
    \item dynamischen input auf Speicherüberlauf testen
    \item Nutzung von Programmiersprachen mit integriertem Schutz (z.B. Java),
      die die Grenzen des Speichers überwachen
  \end{itemize}

\subsubsection{Compiler}
  \begin{itemize}
    \item Überprüfungscode-Erzeugung
    \item Einfügen einer Zufallszahl zwischen Rücksprungadresse und lokalen
      Variablen (wird bei jeden Unterprogrammaufruf eingefügt. Wird diese
      Zufallszahl von Unterprogramm verändert wird der jeweiligen
      Rücksprungaddresse nicht mehr getraut)
    \item Kopie der Rücksprungadresse, diese Adresse wird dann genutzt, was das
      Aussnutzen von Bufferoverflows erschwert
  \end{itemize}
\subsubsection{Betriebssystem}
  \begin{itemize}
    \item  gleichzeitig nicht mehr Speicher zur Verfügung gestellt wird, als der
      lineare Adressraum groß
    \item Nullteterminierung des Zielstrings (OpenBSD)
  \end{itemize}


\section*{Aufgabe 4 - ROP}



\end{document}
